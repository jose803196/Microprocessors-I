\section{Introducción}

El tema de este proyecto es el desarrollo de un generador de ondas senoidales utilizando el microcontrolador PIC18F45K50. Este dispositivo estará compuesto por un teclado, una pantalla y un convertidor digital-analógico (DAC) para la producción de señales senoidales con frecuencias específicas. Este trabajo se realiza para aplicar los conocimientos adquiridos en la asignatura de Microprocesadores I, donde se abordan temas como la programación de microcontroladores, el uso de timers y la gestión de interrupciones. La creación de un generador de señales senoidales es fundamental en diversas aplicaciones electrónicas, permitiendo al estudiante experimentar con la teoría en un entorno práctico. Además, este proyecto estimula el aprendizaje de la integración de hardware y software, y fomenta habilidades en el manejo de sistemas embebidos, así como el diseño de interfaces de usuario.\\

Estará diseñado en varias etapas: se comenzará con la configuración del DAC para generar la señal senoidal. Posteriormente, se desarrollará la interacción mediante un teclado y se implementará la visualización de la frecuencia en un display. Para esto, se utilizarán la tarjeta de enseñanza MEPIC y la tarjeta de interfaz de teclado y pantalla, empleadas en prácticas anteriores. Finalmente, se incorporará una comunicación serie a través del USART, que permitirá la recepción de órdenes y el envío de información sobre la frecuencia actual.El método utilizado en este trabajo se basa en la programación modular(dividir en subprogramas) y el uso de interrupciones. Se implementará el Timer 0 del microcontrolador para generar interrupciones cada 62.5 µs, lo cual se utilizará como frecuencia de muestreo para la señal senoidal. La frecuencia de salida será controlada por una variable global llamada “frec val”, que podrá ser modificada mediante el teclado o a través de comandos recibidos por el puerto USART. Este enfoque garantiza que la señal se genere de forma continua e independiente de otras operaciones del sistema, como la gestión del teclado y la visualización en el display. Además, el sistema se ajustará automáticamente a cualquier cambio en la variable de frecuencia en cualquier momento del funcionamiento.